\documentclass[twoside,german]{elsarticle}
\usepackage[T1]{fontenc}
\pagestyle{headings}
\usepackage{amsthm}

\makeatletter
\theoremstyle{plain}
\newtheorem{thm}{\protect\theoremname}

\makeatother

\usepackage{babel}
\providecommand{\theoremname}{Aufgabe 1}

\begin{document}

\begin{frontmatter}{}
    \title{Datenkommunikation, Aufgabe 1}

    \author{Fabian Hammerschmidt}
    \author{Matthias Schugg}
    \author{Christian Fröhlich}
    \author{Felix Golatofski}

    \begin{abstract}
        Alle hier gezeigten Dokumente sind nur zur Nutzung im Rahmen der Teilnahme an der Datenkommunikations-Vorlesung der Fakultät Informatik an der Hochschule Augsburg freigegeben. Die Angaben der Aufgaben sind urheberrechtlich geschützt!
    \end{abstract}
\end{frontmatter}{}

\section{Aufgabe 1}
\begin{enumerate}[a.]
    \item L/R (Geringere Bandbreite dominiert)
    \item N * L/R
    \item L/R; N * L/R
\end{enumerate}

\end{document}